\documentclass{article}
\usepackage[utf8]{inputenc}
\usepackage{authblk}
\usepackage{amsmath}

\title{\bf Two dimensional discrete scan statistics with variable window shape}

\author[1,2]{Alexandru Am\u{a}rioarei}
\author[3]{Micha\"el Genin}
\author[4,5]{Cristian Preda}

\affil[1]{Faculty of Mathematics and Computer Science, University of Bucharest, Romania}
\affil[2]{National Institute of Research and Development for Biological Sciences, Bucharest, Romania}
\affil[3]{Faculté de Médecine, EA2694, Université de Lille, France}
\affil[4]{Laboratoire Paul Painlevé, Université de Lille, France}
\affil[5]{Institute of Mathematical Statistics and Applied Mathematics of the Romanian Academy, Bucharest, Romania}

\date{December 2018}

\begin{document}

\maketitle

\begin{abstract}
The definition of the the two-dimensional discrete scan statistic with rectangular window shape is extended to a more general framework in which the scanning window is constructed based on a score function. In particular, this approach allows to introduce different shapes for the scanning window (discretized rectangle, polygon, circle, ellipse or annulus). We provide approximation for the distribution of the scan statistic and illustrate their accuracy by conducting a numerical comparison study. The power of test based on the scan statistics is also evaluated by simulation.  
  
\end{abstract}

\section{Introduction}


Let $T_1, T_2\geq 2$ be positive integers, $\mathcal{R}=[0, T_1]\times[0, T_2]$ be a rectangular region and $X_{s_1,s_2}$, $1\leq s_j\leq T_j$, $j\in\{1,2\}$, be an array of independent and identically distributed random variables from a specified distribution $G$ (Bernoulli, binomial, Poisson, normal, etc.) associated to the elementary sub-region $r(s_1,s_2) = [s_1-1,s_1]\times[s_2-1,s_2]$. The two dimensional scan statistics introduced by \cite{Chen1996} is defined as the largest number of events in any rectangular scanning window of size $m_1\times m_2$, where $1\leq m_1\leq T_1$ and $1\leq m_2\leq T_2$ are positive integers, within the rectangular region $\mathcal{R}$, i.e. 

\begin{equation}\label{eq1}
   S_{m_{1},m_{2}}(T_{1}, T_{2}) =\max_{\substack{1\leq i_1\leq T_1-m_1+1 \\ 1\leq i_2\leq T_2-m_2+1}}{Y_{i_1,i_2}}=\max_{\substack{1\leq i_1\leq T_1-m_1+1 \\ 1\leq i_2\leq T_2-m_2+1}}{\sum_{s_1=i_1}^{i_1+m_{1}-1}\sum_{s_2=i_2}^{i_2+m_{2}-1}X_{s_1,s_2}} 
\end{equation}
\noindent
In this paper we present a natural extension of the classical definition of the two dimensional scan statistics by taking in Eq.~\ref{eq1}, instead of a moving sum, a moving \textit{score} evaluated with the help of a function applied on blocks of random variables. 


%%%%%%%%%%%%%%%%%%%%%%%%%%%%%%%%%%%%%%%%%%%%%%%%%%%%%%%%%%%%%%%%%%%%%%%%%%%%%%%%%%%%%%%%
%%%%%%%%%%%%%%%%%%%%%%%%%%%%%%%%%%%%%%%%%%%%%%%%%%%%
%%%%%%%%%%%%%%%%%%%%%%%%%%%%%%%%%%%%%%%%%%%%%%%%%%%%

\bibliographystyle{apalike}
\bibliography{References/Reference_Chap_ScanGeom}

\end{document}
